%----------------------------------------------------------------------------------------
%----------------------------------------------------------------------------------------
%----------------------------------------------------------------------------------------
%Discussion
%----------------------------------------------------------------------------------------
%----------------------------------------------------------------------------------------
%----------------------------------------------------------------------------------------

\section{DISCUSSION}
    \subsection{What we learned from smaller size map}
    \subsection{What we learned from bigger size map}
    Fig.~\ref{fig: all_derived_ones} shows the SOM and the weight plane of each input data from Table~\ref{tab: derived_ones} which is results from already reduced data.
    We kept derived values and values that had not been used to create any of derived values.
    
    Regions 4 and 7 are in the top left of Fig.~\ref{fig: all_derived_ones} and in the second neighbourhood of each other, with very bright light between their nodes which represents that weight of these nodes are similar to each other. 
    The positions of these two regions are really close and they are on the same part of the galaxy.
    Both of these regions are right on the edge of the star forming rings in the galaxy. 
    Position of the region 3 on the SOM is close to the position of the     regions 4 and 7, but with darker colour between the nodes. 
    This region in M31 is close to regions 4 and 7 but on the other side of the star forming ring.
    
    Regions 5, 8 and 6 are close to each other on the SOM.
    These three regions are around the inner ring of the galaxy.
    Since region 5 is right outside the inner rings whereas regions 6 and 8 are in inner part of the inner ring, this region has more distance and more weight from the regions 6 and 8.
    Regions 1 and 9 are close to each other and in the end side of the star forming ring. 
    
    Region 2 is more distant from other regions in the galaxy. 
    However, it is right out side the star forming too. 
    Therefore, its place in SOM somehow is close to the positions of the regions 1,3 and 9.
    The weights between region 2 and the other three regions are relatively high, due to fact that the region 2 is basically placed on the opposite side of the galaxy. 
    Region 10 placed in the bulge of the galaxy, and its position on the SOM is isolated from all the other regions. 

    \subsection{CAVEATS}