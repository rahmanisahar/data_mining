%----------------------------------------------------------------------------------------
%----------------------------------------------------------------------------------------
%----------------------------------------------------------------------------------------
%Results
%----------------------------------------------------------------------------------------
%----------------------------------------------------------------------------------------
%----------------------------------------------------------------------------------------
\section{RESULTS}
%% General things about results
    One of the features of clustering methods is that users must decide what number of clusters is ``sufficient" for their data.
    To do so, we created SOMs with various grids to find ``sufficient size" of map for our data to be $1\times2$ and $10\times10$, in the case of 1D and 2D, respectively. 
    Since in 1D maps each neurons only has maximum two immediate neighbours, they are useful to get general pictures of the data.
    In contrast with 1D maps, 2D maps provide the wining neurons enough space to be in contact with maximum six other neurons; thus give us more detailed picture of clustered data and the topology of our data.
    1D maps are good for studying the general behaviour of our data, while 2D maps shows more details.
    
    %To test the networks we used M101 data to validate data.
%% Results of the 1D map. 
    \subsection{SOMs of M31 data}
    \subsubsection{1D SOMs}
     %%%In this section, we are going to talk about How data behave from 1x2;1x3 and maybe 1x14 networks
      
       